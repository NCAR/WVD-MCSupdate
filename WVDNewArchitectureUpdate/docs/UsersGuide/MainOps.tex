
\chapter{MainOps}
\label{CH-Ops}

This is the most fundamental part of the software. While deployed in the field this is the mode that the software will be running in for a large majority of the campaign. To begin main operations click the MainOps button. It will take a moment for all children to start but when they do you should see all children as responsive on the right hand side. The tab names are configurable via the Configure\_WVDIALMainOps.txt configure file. I will refer to them by their default names. 

The first tab is always Main Controls, which holds the controls to begin modes of operation. The next two are Raw Data and MCS Controls which are used to communicate with the MCS and gather photon counting and power data. The Laser Locking tab controls the wavemeter and is used to control the lasers and etalons. The UPS monitors the state of power to the unit and the UPS battery. The Adv. Visualization tab is used to display composite variables, but does not write any data. This tab could completely crash and would not affect the final data products in any way. The NetCDF tab is used to control the python scripts that write out the final and intermediate data products. Lastly the Weather Station tab controls the weather station. 

When you start main operations be sure that all children are reporting as responsive to the main container, then flip through the tabs to be sure each is responding as you would expect. For the Raw Data tab make sure that photon returns look reasonable, that peaks are showing up where clouds are present, that the data is changing on the expected interval (1/2 Hz by default), etc.... Go into the Laser Locking tab and make sure the wavemeter is responsive. There is a known problem with the wavemeter that causes it to return a default value instead of actual wavelengths, so make sure to give this 5-15 seconds for the wavemeter to respond with real values. If the unit is running not on the battery then check that the UPS is reporting a full battery, and if it is not then make sure the battery is charging and report the UPS behaviour. Check the Weather Station tab to ensure that the weather station is responsive with physically sensable values, the default frequency for weather station returns is 1/10 Hz so don't be suprised if it looks like it isn't updating. Lastly the NetCDF tab will begin by running the python on startup. This python code can take a minute to run so if the output and error boxes are empty then check that the PythonRunning light is lit. When the python is done running the output and error fields should be filled. If the python runs sucessfully the end of the output box should end with a Goodnight World line that has a time stamp. The error field is not empty even in the event of sucess. Seeing output in this field is not a problem or a bug, it should look like a series of paths to output files. If there is any error reporting in this field that is not a path to an output file, then that is an error, and needs attention. Attempts were made to log errors and warnings and the location of those error and warning files is printed at the top of the output. 

The data should be written to the Data directory, as well as being RSynced to an external hard drive connected to the unit, and lastly it should be RSynced to Eldora. Make sure that the data product is actually being written to each of these locations. 

If you've looked at all these tabs and checked the data output then congratulations, you're done. You can walk away and let the DIAL unit run for as long as needed. You can collect your final data products as described in chapter~\ref{CH-Data}.

